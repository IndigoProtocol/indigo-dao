As users and supporters of the Cardano network, the members of the
Indigo community come together to form this decentralized autonomous
organization (DAO) to best serve our community.

The Indigo Protocol and the Indigo DAO have been created to help
decentralized technologies facilitate greater access to financial
services, freedom, and independence within a cooperative spirit of
honesty, integrity, and equality. The Indigo DAO is intended to meet
these ambitious goals by adhering to the following core principles:

\begin{itemize}
\item
  Ensuring that all information shared, and all actions taken, are done
  with the greatest level of transparency possible.
\item
  Ensuring that all actions taken are immutable and subject to publicly
  verifiable audits.
\item
  Providing a forum where all voices are heard, and the expressed will
  of the community is implemented.
\item
  Adherence to the mantra of the Indigo Protocol: Tokenize Everything.
\end{itemize}

\hypertarget{article-i-organization}{%
\section{Article I -- Organization}\label{article-i-organization}}

\textbf{Section 1.1 -- Status.} The Indigo DAO is an informal
non-jurisdictional, non-hierarchical, and nonprofit association of
individuals and entities who agree to act together and who hold the INDY
token, and which functions in part through smart contracts operating on
the Cardano blockchain platform. The Indigo DAO is the steward of the
Indigo Protocol and has the ability to control the application and all
related intellectual property.

The Indigo DAO is not intended to be, and shall not be deemed to be, a
legal entity such as a partnership, joint venture, or other
organization.

\textbf{Section 1.2 -- Purposes.} The Indigo DAO has formed for the
purpose of developing, managing, improving, and maintaining the Indigo
Protocol.

\textbf{Section 1.3 -- Not an Investment Company.} The Indigo DAO shall
not provide financial or investment services, advice, or opportunities.
Any documents, data or other materials developed by the Indigo DAO shall
be placed in the public domain or (as to software) under an appropriate
open-source license; except that personally identifiable information or
other information which is subject to protection under applicable law
shall be excluded from such public publication. Any Indigo DAO projects
or work product are not investments and no promises can be or shall be
made regarding any monetary value or returns of any project or work
product.

\textbf{Section 1.4 -- Cooperative principles.} The Indigo DAO shall be
operated in the spirit of the cooperative principles framed by the 1995
General Assembly of the International Co--operative Alliance, including:

\begin{enumerate}
\item
  voluntary and open participation;
\item
  democratic control through distributed ledger smart contract voting
  management;
\item
  economic participation;
\item
  autonomy and independence of Indigo DAO;
\item
  providing education, training and information;
\item
  cooperation with other DAO cooperatives; and
\item
  concern for the greater community.
\end{enumerate}

\textbf{Section 1.5 -- Nondiscrimination.} The Indigo DAO shall not
arbitrarily discriminate based on race, nationality, religion, age,
gender, sexual orientation, disability, political affiliation, or
otherwise.

\textbf{Section 1.6 -- Term and Continuation of DAO.} The Indigo DAO
will continue perpetually so long as Members do not vote to cease the
activities of the Indigo DAO. Should the Members ever vote to cease
operation of the Indigo DAO, they shall vote to transfer all assets
either (i) to one or more elected Trustee(s) on the conditions stated in
such vote, or (ii) to one or more reputable charitable organizations
selected in line with the ethos and mission of Indigo DAO.

\hypertarget{article-ii-membership}{%
\section{Article II -- Membership}\label{article-ii-membership}}

\textbf{Section 2.1 -- Eligibility.} Membership shall be voluntary and
application open to any person or entity that: (a) holds the private
keys controlling any INDY token staked to the Governance Contract; (b)
seeks to contribute to, build on, or use the services of Indigo DAO, and
(c) agrees to and accepts the responsibilities and terms of this
Constitution. Each person or entity meeting the above criteria is a
``\textbf{Member}.'' There shall be no requirement to contribute time to
the Indigo DAO projects or assets to the Indigo DAO Treasury to be a
Member.

\textbf{Section 2.2 -- Participation and Rights.} Members may
participate in votes on Indigo DAO proposals generated using the
Governance Contract (\textbf{``Proposal''}) by staking some or all of
their INDY in the Governance Contract. The Governance Contract shall be
the exclusive method of holding and recording such votes of the Indigo
DAO Members. The Indigo DAO may also utilize the Governance Contract to
administer and facilitate certain other arrangements and transactions
involving the Indigo DAO, Members and/or third parties.

The Governance Contract shall contain the rules of the Indigo DAO
governance to act on behalf of the Indigo DAO and to contract on behalf
of the Indigo DAO through Indigo DAO membership vote. Indigo DAO Members
will cast votes and carry out the decisions made via the Governance
Contract.

\textbf{Section 2.3 -- Responsibilities.} Members are responsible for
deciding the priorities of the Indigo DAO. Members agree to use the
products or services of the Indigo Protocol on at least an occasional
basis and to stay reasonably current on the state of the DAO (by
reviewing correspondence from the
\href{https://forum.indigoprotocol.io/}{Indigo DAO Forum} (the
``\textbf{Forum}'') or otherwise).

\textbf{Section 2.4 -- Limitations.} No Member has any authority, right,
or power to act as an agent, representative or otherwise on behalf of
the Indigo DAO or any other Member; to bind the Indigo DAO or any other
Member in any way (including by way of Agreement or Liability); or to
convey any Indigo DAO Property, asset, or other right of the Indigo DAO
or any Member. Without limiting the generality of the foregoing, no
Member shall be deemed the partner of the Indigo DAO or any other
Member; and no Member shall state, purport, imply, hold out or represent
to any person that such Indigo DAO Member or any other Indigo DAO Member
has any such authority, right, or power.

To the maximum extent permitted by applicable law, no Member shall be
liable in any way to any person or entity for any action or inaction of
the Indigo Protocol, the Indigo DAO, or any other Member.

\textbf{Section 2.5 -- Access to information.} Members may access
information concerning all activities of the Indigo Protocol and the
DAO, and communicate regarding governance matters, via the Forum, as
well as social media or communications platforms that may be designated
or created in the future by the Indigo DAO or Core Contributors.

\textbf{Section 2.6 -- Settlement of disputes.} All Members agree that
any dispute between Members which cannot be resolved through informal
negotiation shall be resolved by mediation before an impartial mediator
mutually agreed by the involved Members. Members further agree to
utilize decentralized dispute mechanisms via smart contract protocols to
resolve any dispute if such mechanisms are available.

\textbf{Section 2.7 -- Withdrawal.} A Member may withdraw at any time by
unstaking all INDY which they control from the Governance Contract.

\hypertarget{article-iii-decentralized-governance}{%
\section{Article III -- Decentralized
Governance}\label{article-iii-decentralized-governance}}

\textbf{Section 3.1 -- Powers and duties.} All Indigo DAO powers shall
be exercised by or under the authority of Members or such agents,
vendors, contractors, or other designees approved by Members.

\textbf{Section 3.2 -- Voting Structure.} Voting will be performed on
Cardano according to the mechanisms established by the Governance
Contract. Records of votes shall be recorded on Cardano.

\textbf{Section 3.3 -- Voting Procedures}. Voting processes and
procedures applicable to the Indigo DAO are set forth in the Indigo
Decentralized Governance Processes and Procedures available on the Forum
(the ``\textbf{DGPP}''). The DGPP may be supplemented, revised, and
amended by vote of the Members through a successful Proposal.

\textbf{Section 3.4 -- Voting Quorums and Thresholds.} A ``Quorum
Threshold'' is the ratio of votes relative to the total possible votes.
The Quorum Threshold is determined for each Proposal according to
Adaptive Quorum Biasing (\textbf{``AQB''}). ABQ algorithmically sets the
``Yes'' votes required based on the total number of votes relative to
the total number of possible votes. AQB ensures that when there's a low
voter turnout then a larger number of those votes must be ``Yes.''
Conversely, when the voter turnout is high then the number of ``Yes''
votes required can be lower. ``Yes'' votes must always outnumber ``No''
for the Proposal to be successful. The formula for AQB is described
further in the \href{https://indigoprotocol.io/whitepaper}{Indigo
Whitepaper}.

\textbf{Section 3.5 -- Core Contributors.} ``Core Contributor'' means a
Member elected by other Members to be on one or more advisory groups of
persons or entities who are tasked with the responsibility to manage the
administrative and technical operations of Indigo Protocol, the
Governance Contract, or the Indigo DAO, organizing meetings of Members,
submitting governance Proposals, or leading Working Groups.

Core Contributor Members shall serve in the role until the Membership
votes to remove them from the role or they resign. New Core Contributors
may be approved via a vote of the Members through a successful Proposal.

\textbf{Section 3.6 -- Working Groups.} ``Working Group'' means a small
unit of Members who are selected by Core Contributors or Members to
engage in specific projects or tasks designated by the Membership.
Working Group Members shall serve in the role until the Working Group
has completed its project or task, until resignation, or until the
Membership votes to remove them from the Working Group.

\textbf{Section 3.7 -- Moderators.} ``Moderator'' means a Member who is
elected by the Membership to administer the communications,
administrative procedures, and technical operations applicable to the
Indigo DAO, as set forth herein or in the DGPP. Moderators shall serve
in the role until resignation or the Membership votes to remove them
from the Moderators.

\hypertarget{article-iv-meetings-of-members}{%
\section{Article IV -- Meetings of
Members}\label{article-iv-meetings-of-members}}

\textbf{Section 4.1 -- Meetings.} Meetings of Members may be scheduled
either by Core Contributors, or the Moderators where sufficient support
is reflected in the Indigo DAO communication platforms, or by Members
via a successful vote. The purpose of any meeting should be clearly
articulated.

\textbf{Section 4.2 -- Time and place.} The date, time and place of any
meeting shall be determined by the Moderators or Core Contributors
setting the meeting.

\textbf{Section 4.3 -- Notice.} Notices of meetings shall be posted on
the Forum and on Indigo governance communication platforms by the
Moderators. Any business conducted at a meeting other than that
specified in the notice of the meeting shall be of an advisory nature
only.

\hypertarget{article-v-relationships}{%
\section{Article V -- Relationships}\label{article-v-relationships}}

\textbf{Section 5.1 -- Purpose.} To further the objectives of this
Constitution, the Indigo DAO may establish a commercial or business
relationship between the Indigo Protocol or Indigo Foundation and other
projects, persons, or entities (each a ``\textbf{Relationship}'').

\textbf{Section 5.2 -- Negotiating.} Unless this Section 5.2 is
expressly modified in an approved Proposal to establish a Relationship,
each such Proposal shall be deemed to authorize the Indigo Foundation to
negotiate the terms of and execute such agreements or other documents as
may be deemed necessary or advisable by the Indigo Foundation members,
acting in good faith and exercising their fiduciary duty to the Indigo
DAO, to carry out the intent and accomplish the purposes of the Proposal
(each such final written agreement a ``Relationship Agreement'').

Relationship Agreement negotiations are expected to be maintained as
confidential until such time as a final written agreement is executed by
the Indigo Foundation and the other project, person, or entity.

If any Relationship Agreement calls for the creation of projects, tools,
or products, the terms shall provide that the parties place any related
software code under an open license unless the authorizing Proposal
directs otherwise.

\textbf{Section 5.3 -- Relationship Limitations.} The Members may,
through a vote of the Indigo DAO, terminate, withdraw from, or amend any
Relationship in accordance with the terms of the Relationship Agreement.

\hypertarget{article-vi-fiscal-matters}{%
\section{Article VI -- Fiscal Matters}\label{article-vi-fiscal-matters}}

\textbf{Section 6.1 -- No Patronage Dividends.} There will be no
patronage dividends offered to Indigo DAO Members.

\textbf{Section 6.2 -- Use of DAO Treasury}. Assets will be distributed
from the DAO Treasury only for the purpose of:

\begin{itemize}
\item
  Grants of tokens to persons or entities that make contributions (e.g.,
  distributions by operation of the Indigo Protocol, code submissions,
  bug bounties, community grants, etc.) which further the objectives of
  the Indigo Protocol;
\item
  Payments under Relationship Agreements; or
\item
  Other grants or payments specifically authorized by successful
  Proposal.
\end{itemize}

\textbf{Section 6.3 -- No Redemption Rights.} Members have no rights of
redemption of any working capital, or token assets, controlled by the
Indigo DAO.

\hypertarget{article-vii-indigo-foundation}{%
\section{Article VII -- Indigo
Foundation}\label{article-vii-indigo-foundation}}

\textbf{Section 7.1 -- Indigo Foundation Appointment.} The Indigo DAO
hereby designates and appoints the Indigo Foundation to serve as the
sole legal representative of the Indigo DAO. The Indigo Foundation is a
not-for-profit Cayman Islands entity. The Indigo Foundation will have a
minimum of three Board of Director Members.

The Indigo Foundation's charter confirms that it is obligated to, acting
in good faith and in the best interests of the Indigo DAO, and safeguard
the interests of the Indigo Protocol and support the Indigo DAO. The
Indigo Foundation shall have the power to enter into binding
Relationship Agreements as directed by a successful Proposal passed by
the Members, as set forth in Article V.

\textbf{Section 7.2 -- Indigo Foundation Formation; Charter; Initial
Foundation Board Members.} The Indigo Foundation will be a
not-for-profit Cayman Islands foundation company (registration filed
September 22, 2022; registration number pending). The Indigo Foundation
shall hold all the rights and liabilities of the Indigo Protocol and
Indigo DAO, including all intellectual property interests and property
rights. The Indigo Foundation is authorized to act to implement and
enforce all Relationship Agreements and protect any legal or property
rights of the Indigo DAO.

\textbf{Section 7.3 -- Foundation Board Member Elections.} Foundation
Board of Director members may resign at any time and may be removed at
any time by a vote of the Members. A majority vote of the Members is
required to appoint a candidate to an open Board seat; re-appoint any
incumbent Board member for an additional term; or expand the number of
Board members (providing that the number of approved Board member seats
must always be odd). Votes should be scheduled to ensure that all
authorized Board seats are always occupied.

If no single candidate for a Board member position receives successful
approval by a vote of Members for any open Board seat, a runoff vote
will be held within two days after certification of the prior results
between the two candidates who received the greatest number of votes in
the prior vote.

\textbf{Section 7.4 -- Indigo Foundation and Board Member
Indemnification.} The Indigo DAO shall indemnify, defend and hold
harmless the Indigo Foundation, its Board members, and any managers,
officers, employees, from and against any and all loss, costs,
penalties, fines, damages, claims, expenses (including attorney's fees)
or liabilities to the fullest extent possible, arising out of, resulting
from, or in connection with the actions of the Indigo Foundation, to the
fullest extent of the law; except the foregoing shall not apply to the
extent that any Liability arises from the fraud or intentional
misrepresentations of the Indigo Foundation or any individual Board
member. The Indigo DAO shall purchase for the Indigo Foundation such
insurance as may be available and recommended, including but not limited
to directors' and officers' liability insurance and general liability
insurance.

\hypertarget{article-viii-miscellaneous}{%
\section{Article VIII --
Miscellaneous}\label{article-viii-miscellaneous}}

\textbf{Section 8.1 -- Interpretation.} The Core Contributors shall have
the power to interpret this Constitution, apply its provision to
particular circumstances, and adopt policies in furtherance of them, and
such interpretations shall be presumed correct unless superseded by
successful Proposal.

\textbf{Section 8.2 -- Severability.} In the event that any provision of
this Constitution is determined to be invalid or unenforceable under any
statute or rule of law, then such provision shall be deemed inoperative
to such extent and shall be deemed modified to conform with such statute
or rule of law without affecting the validity or enforceability of any
other provision of this Constitution.

\textbf{Section 8.3 -- Amendment.} This Constitution may be amended by a
successful Proposal via the Governance Contract. Upon approval by the
Members, verification of the amended Constitution will be recorded on
the Cardano blockchain.

\textbf{Section 8.4 -- Communication by electronic means.} Unless
otherwise required by law or by agreement, any notice, vote, consent,
petition, or other oral or written communication required or permitted
can be delivered by electronic means; provided that, in the case where
such communication expressly or impliedly requires the signature of a
person, means are in place to reasonably assure the authenticity of the
signature.

\hypertarget{article-ix-additional-definitions}{%
\section{Article IX -- Additional
Definitions}\label{article-ix-additional-definitions}}

\textbf{``Governance Contract''} means the suite of Indigo DAO
governance related smart contracts deployed to the Cardano blockchain.

\textbf{``Agreement''} means any: (i) written, oral, implied by course
of performance or otherwise or other agreement, contract, understanding,
arrangement, settlement, instrument, warranty, license, insurance
policy, benefit plan or legally binding commitment or undertaking; or
(ii) any representation, statement, promise, commitment, undertaking,
right or obligation that may be enforceable, or become subject to a
Legal Order directing performance thereof, based on equitable principles
or doctrines such as estoppel, reliance, or quasi-contract.

\textbf{``Indigo DAO Property''} means any token or other asset, right
or property licensed to or on deposit with or owned, held, custodied,
controlled or possessed by or on behalf of the Indigo DAO, including any
token on deposit with or held, controlled, or possessed by the
Governance Contract or any associated smart contracts related to the
Indigo Protocol.

\textbf{``Legal Order''} means any restraining order, preliminary or
permanent injunction, stay or other order, writ, injunction, judgment,
or decree that either: (i) is issued by a court of competent
jurisdiction, or (ii) arises by operation of applicable law as if issued
by a court of competent jurisdiction, including, in the case of clause
``(ii)'' an automatic stay imposed by applicable law upon the filing of
a petition for bankruptcy.

\textbf{``Liability''} means any debt, obligation, duty or liability of
any nature (including any unknown, undisclosed, unmatured, unaccrued,
unasserted, contingent, indirect, conditional, implied, vicarious,
inchoate derivative, joint, several or secondary liability), regardless
of whether such debt, obligation, duty or liability would be required to
be disclosed on a balance sheet prepared in accordance with generally
accepted accounting principles and regardless of whether such debt,
obligation, duty or liability is immediately due and payable. To be
``Liable'' means to have, suffer, incur, be obligated for or be subject
to a Liability.
