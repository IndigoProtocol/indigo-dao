\textbf{Adopted Pursuant to Article III, Section 3.3 of the Indigo DAO
Constitution}

The following Decentralized Governance Processes and Procedures (the
``Procedures'') are the proposal creation and voting processes and
procedures that shall apply to any vote of the Indigo DAO pursuant to
the Indigo DAO Constitution (the ``Constitution'').

\hypertarget{a.-vote-forums-and-information-sources}{%
\section{A. Vote Forums and Information
Sources}\label{a.-vote-forums-and-information-sources}}

The following sources and venues are adopted to provide Members
information regarding the Indigo DAO and all Proposals:

\begin{enumerate}
\item
  \url{https://indigoprotocol.io/}~\\
  The primary Indigo website with information regarding the protocol's
  features and functions, and why it was built.
\item
  \url{https://forum.indigoprotocol.io/} (the ``Forum'')\\
  The communication platform for Members and others to engage with the
  Indigo community regarding various Temperature Checks, Proposals, and
  announcements. Prior votes of Members on Proposals will be available
  here.
\item
  \url{https://github.com/IndigoProtocol}~\\
  The location where implemented or proposed script or code consisting
  of technical Indigo software development information can be shared and
  viewed.
\item
  \url{http://discord.gg/indigoprotocol}~\\
  The primary communication platform where the larger Indigo community
  can socialize, discuss all aspects of the protocol and the larger defi
  community, and learn about the community from the community itself.
\end{enumerate}

\hypertarget{b.-proposal-creation}{%
\section{B. Proposal Creation}\label{b.-proposal-creation}}

The following sections detail the progressive steps that one or more
Members (as defined in Section E below) must follow to present a new
idea or suggestion to the Indigo community for a vote. Moderators will
be available to assist the community and Members with the mechanics of
governance in accordance with these Procedures.

\hypertarget{step-1-temperature-check}{%
\subsection{Step 1: Temperature Check}\label{step-1-temperature-check}}

The Temperature Check is a way for one or more community members to
share a proposed idea or suggestion and receive initial feedback from
the community. One does not have to be a Member to create or participate
in a Temperature Check.

Through community feedback and open discourse, the submitting community
members can craft a more informed proposed idea or suggestion and
thereafter submit the successful Temperature Check as a Proposal to be
voted on by Members.

To create a Temperature Check:

\begin{enumerate}
\item
  Ask a general question to the community in the
  \href{https://forum.indigoprotocol.io/c/general-discussion/8}{``Temp
  Check Discussion''} category of the
  \href{https://forum.indigoprotocol.io/}{Forum}, and post a link to the
  Forum and short summary of the proposal to the
  \href{https://discord.com/channels/816779565796032513/934082020454309888}{Indigo
  DAO governance channel} in
  \href{https://discord.gg/indigoprotocol}{Discord}. For example, one
  may have an idea or proposal to green light the creation of a new
  synthetic asset, delist an existing synthetic asset, or add a new
  feature to benefit the Indigo protocol.

  If your proposed idea involves the whitelisting of a new synthetic
  asset, you must identify one or more pricing data sources which are or
  can be provided to an Indigo-supported Oracle service provider.

  If your proposed idea involves significant changes to the protocol,
  you should first start a conversation with the Core Contributors to
  discuss the technical aspects of the idea.

  Be engaged with the discussion and be willing to evolve your proposal
  based on the community feedback, as history shows that many initial
  ideas or proposals are materially improved by the community.
\item
  After you have a vetted and final proposed idea that has considered
  feedback from the community, create a poll in the
  ``\href{https://forum.indigoprotocol.io/c/polls/11}{Temp Check
  Polls}'' category of the Forum using the ``Build A Poll'' feature so
  that others can indicate their interest in the proposal. Please
  include reference to any prior related Forum discussions, issues,
  Proposals, a summary for the proposed idea, and any prior discussions
  on the topic. This poll may be binary or multiple choice but must
  include a ``No'' / ``Make no change'' option.
\item
  Reach out to your personal network, social media network, and Indigo
  DAO Members in order to build support for the proposal. Be actively
  discussing the proposal and be willing to respond to questions posted.
  Please share your reasoning and remember that everyone's objective is
  to further the philosophy and mission of the Indigo DAO. Disagreements
  are healthy indicators of the Indigo DAO functioning as a democracy to
  generate the best possible outcomes for the community and protocol.
\end{enumerate}

At the end of the poll, the option with the majority of the votes wins.
If ``Yes'' passes, the idea can proceed to Step 2 and be presented as a
formal Governance Proposal.

If a Temperature Check poll does not get a majority of ``Yes'' votes and
the community members seek to submit a revised proposed idea that takes
into account the concerns raised, this Step 1 process may be repeated
with each successive related post adding `v2, v3, etc.' to the title of
the revised ideas.

The content of a Temperature Check should not be changed after the
Temperature Check poll has begun to ensure the integrity of the poll
results. Moderators will flag any breach of this requirement.

\hypertarget{step-2-governance-proposal-governance-portal}{%
\subsection{Step 2: Governance Proposal -- Governance
Portal}\label{step-2-governance-proposal-governance-portal}}

A formal on-chain Governance Proposal (a ``Proposal'') is an idea or
suggestion that has received sufficient support in the Temperature
Check, and that one or more Members now wish to submit for a governance
vote by Members. A Proposal must be materially identical as the
successful Temperature Check. A Proposal includes only two voting
options: ``Yes'' and ``No.''

The Member(s) creating a Proposal must deposit the amount of INDY
required by the ``Proposal Deposit'' protocol parameter in order to
create the Proposal. Upon submission of a Proposal, the deposited INDY
is held in custody by the protocol. Any Proposal which is not successful
will result in the Proposal Deposit being distributed to the DAO
Treasury.

After submitting the Proposal, the creating Member must create shards
for the Proposal by depositing ADA and submitting transactions. The
number of shards that must be created is defined by the ``Total Shards''
protocol parameter. The amount of ADA that must be deposited per shard
is defined by the Cardano blockchain. Upon creation of each shard, the
deposited ADA is held in custody by the protocol. After the Proposal's
Voting Period has ended, the creating Member can retrieve their
deposited ADA by closing the Proposal.

Follow the applicable steps below to create a Proposal via the
Governance Dashboard in the Indigo Web App.

\hypertarget{text-proposal}{%
\subsubsection{Text Proposal}\label{text-proposal}}

A ``Text Proposal'' is one suggesting text that should be adopted by the
Indigo DAO, such as this Procedures document or the Constitution, or to
appoint or remove Moderators or Core Contributors.

\begin{enumerate}
\item
  Click on the ``Create a Proposal'' or similar link in the Governance
  Dashboard in the Indigo Web App.
\item
  Link the Proposal to an existing successful Temperature Check poll
  from the ``\href{https://forum.indigoprotocol.io/c/polls/11}{Temp
  Check Polls}'' category of the Forum.
\end{enumerate}

Upon submitting a Text Proposal, a new post will be created in the Forum
under the
``\href{https://forum.indigoprotocol.io/c/proposals/6}{On-chain
Proposals}'' category highlighting details about the active Proposal.

\hypertarget{whitelist-iasset}{%
\subsubsection{Whitelist iAsset}\label{whitelist-iasset}}

A ``Whitelist iAsset'' Proposal is one suggesting that a new iAsset
(synthetic asset) be supported by the protocol.

\begin{enumerate}
\item
  Click on the ``Create a Proposal'' or similar link in the Governance
  Dashboard in the Indigo Web App.
\item
  Link the Proposal to an existing successful Temperature Check poll
  from the ``\href{https://forum.indigoprotocol.io/c/polls/11}{Temp
  Check Polls}'' category of the Forum.
\item
  Write the name of the iAsset to create.
\item
  Define the proposed Minimum Collateral Ratio (``MCR'') for the iAsset.
\item
  Input an existing secure, reliable, and Indigo-supported Oracle data
  price feed for the tracked asset.
\end{enumerate}

Upon submitting a Whitelist iAsset Proposal, a new post will be
submitted to the Forum under the
``\href{https://forum.indigoprotocol.io/c/proposals/6}{On-chain
Proposals}'' category highlighting details about the active Proposal.

\hypertarget{update-iasset}{%
\subsubsection{Update iAsset}\label{update-iasset}}

An ``Update iAsset'' Proposal is one that changes an existing supported
iAsset, such as by updating its Oracle data price feed or its MCR.

\begin{enumerate}
\item
  Click on the ``Create a Proposal'' or similar link in the Governance
  Dashboard in the Indigo Web App.
\item
  Link the Proposal to an existing successful Temperature Check poll
  from the ``\href{https://forum.indigoprotocol.io/c/polls/11}{Temp
  Check Polls}'' category of the Forum.
\item
  Select the iAsset to update.
\item
  If updating the Oracle for the iAsset, input an existing reliable
  Oracle data price feed that can be used for the asset.
\item
  If updating the MCR for the iAsset, input a new MCR value.
\end{enumerate}

Upon submitting an Update iAsset Proposal, a new post will be submitted
to the Forum under the
``\href{https://forum.indigoprotocol.io/c/proposals/6}{On-chain
Proposals}'' category highlighting details about the active Proposal.

\hypertarget{delist-iasset}{%
\subsubsection{Delist iAsset}\label{delist-iasset}}

A ``Delist iAsset'' Proposal is one suggesting that an existing
supported iAsset should be no longer supported by the protocol; upon
approval such iAsset should no longer be mintable by users.

\begin{enumerate}
\item
  Click on the ``Create a Proposal'' or similar link in the Governance
  Dashboard in the Indigo Web App.
\item
  Link the Proposal to an existing successful Temperature Check poll
  from the ``\href{https://forum.indigoprotocol.io/c/polls/11}{Temp
  Check Polls}'' category of the Forum.
\item
  Select the iAsset to delist.
\item
  Define the last known price of the tracked asset.
\end{enumerate}

Upon submitting a Delist iAsset Proposal, a new post will be submitted
to the Forum under the
``\href{https://forum.indigoprotocol.io/c/proposals/6}{On-chain
Proposals}'' category highlighting details about the active Proposal.

\hypertarget{whitelist-lp-token}{%
\subsubsection{Whitelist LP Token}\label{whitelist-lp-token}}

A ``Whitelist LP Token'' Proposal is one that suggests distributing INDY
as a reward to a specific DEX Liquidity Provider Token that is staked
within the protocol.

\begin{enumerate}
\item
  Click on the ``Create a Proposal'' or similar link in the Governance
  Dashboard in the Indigo Web App.
\item
  Link the Proposal to an existing successful Temperature Check poll
  from the ``\href{https://forum.indigoprotocol.io/c/polls/11}{Temp
  Check Polls}'' category of the Forum.
\item
  Input the Cardano policy ID and asset name of the LP Token to
  whitelist.
\item
  Select the iAsset the LP Token provides liquidity for.
\item
  Select the DEX the LP Token is issued by.
\end{enumerate}

Upon submitting a Whitelist LP Proposal, a new post will be submitted to
the Forum under the
``\href{https://forum.indigoprotocol.io/c/proposals/6}{On-chain
Proposals}'' category highlighting details about the active Proposal.

\hypertarget{delist-lp-token}{%
\subsubsection{Delist LP Token}\label{delist-lp-token}}

A ``Delist LP Token'' Proposal is one that suggests discontinuing
distribution of INDY rewards to a specific DEX token that is staked
within the protocol.

\begin{enumerate}
\item
  Click on the ``Create a Proposal'' or similar link in the Governance
  Dashboard in the Indigo Web App.
\item
  Link the Proposal to an existing successful Temperature Check poll
  from the ``\href{https://forum.indigoprotocol.io/c/polls/11}{Temp
  Check Polls}'' category of the Forum.
\item
  Select the LP Token to delist.
\end{enumerate}

Upon submitting a Delist iAsset Proposal, a new post will be submitted
to the Forum under the
``\href{https://forum.indigoprotocol.io/c/proposals/6}{On-chain
Proposals}'' category highlighting details about the active Proposal.

\hypertarget{upgrade-protocol}{%
\subsubsection{Upgrade Protocol}\label{upgrade-protocol}}

An ``Upgrade Protocol'' Proposal is one that suggests updating the
codebase or parameters of the protocol on the Cardano blockchain.

\begin{enumerate}
\item
  Submit the Proposal via the Cardano blockchain. Any user can submit
  this proposal directly to the Cardano blockchain by interacting with
  the protocol as described in the Indigo Yellow Paper.
\item
  Enter the currency symbols for each script to include in the Upgrade
  Path as defined in the Indigo Yellow Paper.
\item
  Under the
  ``\href{https://forum.indigoprotocol.io/c/proposals/6}{On-chain
  Proposals}'' category of the Forum create a new post titled ``Protocol
  Upgrade {{[}}Version Number{{]}}'', including the following steps:

  \begin{enumerate}
  \item
    Include links to the related Temperature Check and poll results, and
    all discussion threads.
  \item
    Describe high level changes that are included within this protocol
    upgrade.
  \item
    If upgrading code of the protocol, include an audit report from a
    reputable audit company (e.g., a Plutus-certified auditor).
  \item
    If code has been updated, include a link to the GitHub repository
    and the related commit hashes of the code.
  \item
    For each upgraded script, include all of the Validator Hashes as
    well as the related Currency Symbols. These Validator Hashes and
    Currency Symbols should match the compiled scripts from the GitHub
    repository and audit report.
  \end{enumerate}
\end{enumerate}

\hypertarget{c.-general-procedures-applicable-to-all-proposals}{%
\section{C. General Procedures applicable to all
Proposals:}\label{c.-general-procedures-applicable-to-all-proposals}}

\hypertarget{temp-check-or-proposal-withdrawal}{%
\subsection{1. Temp Check or Proposal
Withdrawal}\label{temp-check-or-proposal-withdrawal}}

A Temperature Check may withdrawn at any time. A Proposal may not be
withdrawn once submitted.

\hypertarget{moderator-assistance}{%
\subsection{2. Moderator Assistance}\label{moderator-assistance}}

Moderators are available to assist with developing and submitting
proposals at any stage. Anyone who repeatedly utilizes the Forum for
Proposals which don't meet the foregoing requirements may be temporarily
or permanently banned by the Moderators from creating accessing the
Forum.

\hypertarget{proposal-vote-schedule}{%
\subsection{3. Proposal Vote Schedule}\label{proposal-vote-schedule}}

Once a Proposal has been successfully submitted, it will be placed in
line for a vote. Proposals will be voted on in the order submitted.
Ongoing discussions may take place under the related post in the
``On-chain Proposals'' category of the Forum until the Voting Period has
ended.

\hypertarget{voter-eligibility}{%
\subsection{4. Voter Eligibility}\label{voter-eligibility}}

Only Members are eligible to vote on Proposals. In order to vote on a
Proposal, the Member must link a compatible wallet containing INDY to
the Governance Contract.

\hypertarget{proposal-voting-process}{%
\subsection{5. Proposal Voting Process}\label{proposal-voting-process}}

A Proposal open for voting will appear in the Governance Dashboard. You
may review all current and former Proposals there and see the applicable
voting time windows for pending Proposals. If a Proposal is currently
open for voting it will be shown as active. Clicking on the Proposal
within the Governance Dashboard will display a link to the Proposal's
associated information, documentation, and prior discussion on the
Forum.

To vote on a Proposal, a Member must delegate INDY tokens that they have
already staked to the Governance Contract. A Member may not change or
withdraw a previously cast vote. Delegated INDY will be released after
the Voting Period for the Proposal.

\hypertarget{successful-proposal-execution-delay}{%
\subsection{6. Successful Proposal Execution
Delay}\label{successful-proposal-execution-delay}}

If a Proposal passes successfully, an Execution Delay (defined by a
protocol parameter) will commence prior to the Proposal being eligible
for execution.

\hypertarget{d.-moderator-roles-and-obligations}{%
\section{D. Moderator Roles and
Obligations}\label{d.-moderator-roles-and-obligations}}

\hypertarget{moderator-selection}{%
\subsection{1. Moderator Selection}\label{moderator-selection}}

Moderators will be selected by vote of the Members (using a ``Text
Proposal''). The number of Moderators will be three; the number of
Moderators can be changed by a vote of the Members. Each Moderator will
provide a Forum username and will be listed publicly as a confirmed
Indigo DAO Moderator. Each Moderator will have a one-year term; however,
the initial three Moderators will have terms of 6 months, 9 months, and
12 months, respectively, so that in the future there are always
incumbent Moderators available to help newly selected Moderators
assimilate into the role.

\hypertarget{moderator-duties}{%
\subsection{2. Moderator Duties:}\label{moderator-duties}}

\begin{enumerate}
\item
  To be a fair and neutral administrator of the Indigo governance
  process to help support the growth, reliability, and trustworthiness
  of the Indigo protocol. Unless stated otherwise, all Moderator actions
  should be collectively decided and agreed on by at least two-thirds of
  all Moderators.
\item
  Be the primary source for information regarding the Indigo DAO
  Constitution and these Procedures within the Forum and fulfill the
  Moderator duties set out in these Procedures.
\item
  Communicate with Members pursuing a Temperature Check or Proposal at
  each stage in the process to assist them in complying with these
  Procedures and the Constitution. Promptly review all Temperature
  Checks or Proposals and communicate to the sponsoring Member(s) and/or
  the Indigo community regarding any elements that do not or may not
  comply with the Constitution or these Procedures.
\item
  Moderate the Forum, communicate with Members regarding governance
  matters and proposals, and enforce all community standards within the
  Forum applicable to these Procedures.
\item
  Support the Indigo Foundation as requested to implement any successful
  Proposal or otherwise.
\end{enumerate}

An overall goal applicable to all Moderator duties is to create a
supportive and collaborative environment that encourages participation
in the Indigo governance process to help ensure the greatest Member
participation possible, while also ensuring that the Indigo governance
process functions in accordance with the Constitution, these Procedures,
and any laws that may apply.

\hypertarget{e.-definitions}{%
\section{E. Definitions}\label{e.-definitions}}

The following capitalized terms are defined in the Indigo DAO
Constitution:

\textbf{Member (Const., Art. II, Sec. 2.1)}: A person or entity that (a)
holds the private keys controlling any INDY token staked to the
Governance Contract; (b) seeks to contribute to, build on, or use the
services of Indigo DAO, and (c) is willing to accept the
responsibilities and terms of the Constitution.

\textbf{Governance Contract (Const., Art. IX)}: The suite of Indigo DAO
governance smart contracts deployed to the Cardano blockchain.

The following capitalized terms used herein are defined as follows:

\textbf{Execution Delay:} The time period defined by the ``Effective
Delay'' protocol parameter that must pass before executing a passed
Proposal.

\textbf{Moderator}: An individual who has been selected by the Members
as a neutral administrator to facilitate and manage the governance
process in accordance with the Constitution and these Procedures.

\textbf{Oracle}: A third-party service that feeds external data into the
protocol.

\textbf{Proposal Deposit:} The number of INDY tokens defined by the
``Proposal Deposit'' protocol parameter that a Member must deposit as
collateral in order to open a new Proposal.

\textbf{Vote:} A vote placed through the Governance Contract indicating
either support or rejection for a Proposal. A voter must be a Member and
have INDY staked to the Governance Contract to vote.

\textbf{Voting Period:} The time period defined by the ``Voting Period''
protocol parameter in which Members can vote on a Proposal.
